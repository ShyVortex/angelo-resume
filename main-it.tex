%-------------------------------------
% Resume in LaTeX (LuaLaTeX)
% Author:  ShyVortex
% Project: https://github.com/ShyVortex/angelo-resume
% Based on: https://github.com/Xyz-yuanhf/yuan-resume
%          http://www.jianxu.net/en/
% License: MIT
%------------------------------------

\documentclass[a4paper, 10pt]{article}
\usepackage{myresume}  % Style package

% --------------------  START  --------------------
\begin{document}
	
	% -------------------- HEADING --------------------
	\begin{flushright}
		%\begin{tikzpicture}
		% cerchio esterno (bordo nero)
		%\draw[line width=2pt, black] (0,0) circle (1.50cm); 
		
		% cerchio interno con foto
		%\clip (0,0) circle (1.5cm);
		%\node at (0,0) {\includegraphics[width=3cm]{Figs/pic.jpg}};
		%\end{tikzpicture}
	\end{flushright}\vspace{-45pt}
	
	\begin{flushleft}  % Rplace here with your name (and identity if required)
		{\Calluna \fontsize{27pt}{27pt}\selectfont \textsc{Angelo Trotta}}
		\quad {\Calluna \fontsize{14.5pt}{14.5pt}\selectfont
			\hfill \textsc{Curriculum Vitae}}
		\hspace{0.2cm} \fontsize{10pt}{10pt}{(Ottobre 2025)}
		\noindent\rule{\textwidth}{0.4pt}
	\end{flushleft}
	
	% -------------------- BIO --------------------
	\sectionBlock{
		\section{Bio}
	}{
		\Calluna Laureato in Informatica, appassionato dell’utilizzo della tecnologia per migliorare la vita delle persone. %I’m particularly interested in the intersection of computer science and healthcare, where data-driven innovation and intelligent systems can make a real difference in patient outcomes and medical research.
		\\ Curioso, motivato e attento ai dettagli, amo esplorare nuove tecnologie e sviluppare soluzioni pratiche che avvicinano le persone al mondo digitale.
	}
	
	% -------------------- CONTACTS --------------------
	\sectionBlock{
		\section{Informazioni di Contatto}
	}{
		Residenza Universitaria
		\newline Dipartimento di Bioscienze e Territorio
		\newline Università degli Studi del Molise
		\newline Contrada Fonte Lappone
		\newline 86090 - Pesche (IS), Italia
		
		\setstretch{0.6}
		\skillListStart
		\item \textit{Cellulare: } {\Calluna (+39) 392 775 0987}
		\item \textit{E-mail: } {\url{angelotr@ik.me}}
		\item \textit{PEC: } {\url{a.trotta@spidmail.it}}
		\item \textit{GitHub: } {\url{https://github.com/ShyVortex}}
		\skillListEnd
	}
	
	% -------------------- PERSONAL --------------------
	\sectionBlock{
		\section{Personale}
	}{
		\setstretch{0.6}
		\skillListStart
		\item \textit{Genere: } {Maschio}\vspace{0.25em}
		\item \textit{Data di nascita: } {05/10/2002}\vspace{0.25em}
		\item \textit{Cittadinanza: } {Italiana}
		\skillListEnd
	}
	
	% ---------------- PROFESSIONAL EXPERIENCE ----------------
	\sectionBlock{
		\section{Esperienza Professionale}
	}{
		\internHeading
		{Artista 3D}
		{Libero professionista}{Italia, Remoto}{10/2019 - 12/2022}
		{Responsabile della progettazione creativa e della realizzazione di scene e modelli 3D, combinando visione artistica con competenza tecnica per produrre assets di alta qualità.} \vspace{0.15em}
		\itemListStart
		\myItem{Progettazione e creazione di spazi e modelli 3D.}
		\myItem{Collaborazione a stretto contatto con fotografi e artisti per raggiungere \\ risultati visivi ottimali.}
		\myItem{Sviluppo professionale progressivo nella modellazione 3D e nell'efficienza di utilizzo dei corrispettivi software.}
		\myItem{Esperienza maturata nell'utilizzo di software per modellazione 3D e animazione all'avanguardia, come Blender (1000+ ore) e Unreal Engine.}
		\itemListEnd \vspace{0.5em}
		
		\internHeading
		{Recensore Tirocinante}
		{Devils In The Detail}{Italia, Remoto}{01/2021 - 01/2022}
		{Ho imparato a scrivere recensioni semi-professionali in ambito di videogiochi al fine di informare un eventuale
		acquirente sulle potenzialità, i punti di forza e i punti deboli di ogni videogioco da me recensito.} \vspace{0.15em}
		\itemListStart
		\myItem{Migliorata e approfondita la conoscenza della lingua inglese.}
		\myItem{Pubblicate recensioni di qualità con l'obiettivo di avere sempre una valutazione critica e oggettiva.}
		\myItem{\textbf{Pagina ufficiale curatore:} \url{https://store.steampowered.com/curator/37072886-Devils-in-the-Detail/}}\\
		\myItem{\textbf{Pagina curatore personale:}
			\url{https://store.steampowered.com/curator/40199055/}}
		\itemListEnd \vspace{0.5em}
		
		\internHeading
		{QA Tester}
		{Groove Studio}{Italia, Remoto}{07/2020 - 12/2020}
		{Contribuito al controllo qualità del gioco indie horror in prima persona LOCIS, pubblicato su Steam.} \vspace{0.15em}
		\itemListStart
		\myItem{Ho condotto sessioni di playtesting sistematiche per identificare difetti e problemi con il gameplay.}
		\myItem{Analizzata e valutata la qualità del software per garantire al giocatore un'esperienza fluida e senza intoppi.}
		\myItem{Partecipazione effettiva alla risoluzione di problematiche in team, collaborando anche con gli sviluppatori.}
		\myItem{Ricercate e proposte soluzioni a sfide tecniche e strutturali incontrate durante il testing.}
		\itemListEnd \vspace{0.5em}
		
		\internHeading
		{Content Creator}
		{Maker Studios}{Italia, Remoto}{02/2015 - 02/2018}
		{Gestito un canale YouTube basato sul gaming un tempo attivo con oltre 300 iscritti e diverse migliaia di visualizzazioni. Ho collaborato con una rete di terze parti che garantiva introiti tramite l'integrazione di pubblicità.} \vspace{0.15em}
		\itemListStart
		\myItem{Responsabile della progettazione creativa di video informativi o di intrattenimento.}
		\myItem{Esperienza nell'uso dei principali software di editing video, tra cui Camtasia Studio e DaVinci Resolve}
		\myItem{Implementazione di nuovi processi lavorativi con conseguente risparmio economico, riduzione dello spreco di
		risorse e ottimizzazione del flusso di lavoro}
		\myItem{Collaborazione diretta con colleghi video maker per realizzare contenuti coinvolgenti e coesi.}
		\itemListEnd
	}
	
	% -------------------- EDUCATION --------------------
	\sectionBlock{
		\section{Istruzione e Formazione}
	}{
		\eduHeading
		{\large Laurea Magistrale - Sicurezza dei Sistemi Software}{2026 (previsto)}
		{Università degli Studi del Molise, Pesche (IS)}{} \vspace{0.25em}
		\nbItListStart \myItem{Esami sostenuti finora:} \itemListEnd
		\itemListStart
		\myItem{Advanced English}
		\myItem{Biometric Systems}
		\myItem{Computer Forensics and Investigations}
		\myItem{Security Governance}
		\myItem{Informatics and Law}
		\myItem{Computational Statistics and Machine Learning}
		\myItem{Software Project Management}
		\myItem{Software Security and Program Analysis}
		\itemListEnd
		\vspace{0.5em}
		
		\eduHeading
		{\large Laurea Triennale - Informatica}{24 Ottobre 2024}
		{Università degli Studi del Molise, Pesche (IS)}{} \vspace{0.25em}
		\nbItListStart
		\myItem{Titolo tesi: \href{https://github.com/ShyVortex/analyzer-paper/blob/main/Paper/thesis\_final.pdf}{\textit{A Fine-Grained Analysis of Comments Quality in Code-Related Datasets}}}
		\myItem{Area di ricerca: Automated Software Delivery}
		\myItem{Relatore: Prof. Simone Scalabrino}
		\myItem{Voto finale: 108/110}
		\itemListEnd
		\nbItListStart \myItem{Esami sostenuti:} \itemListEnd
		\itemListStart
		\myItem{Logica e fondamenti di Informatica}
		\myItem{Architettura degli elaboratori}
		\myItem{Programmazione I}
		\myItem{Sistemi Operativi}
		\myItem{Calcolo delle probabiilità e statistica}
		\myItem{Lingua inglese}
		\myItem{Programmazione II}
		\myItem{Matematica I}
		\myItem{Introduction to Machine Learning}
		\myItem{Informatica giuridica}
		\myItem{Algoritmi e strutture dati}
		\myItem{Basi di dati e sistemi informativi}
		\myItem{Fisica}
		\myItem{Calcolo numerico}
		\myItem{Matematica II}
		\myItem{Reti di calcolatori}
		\myItem{Ricerca operativa}
		\myItem{Automated Software Delivery}
		\myItem{Intelligenza Artificiale}
		\myItem{Hands on Unity}
		\myItem{Programmazione mobile}
		\itemListEnd \vspace{0.5em}
		
		\eduHeading
		{\large Diploma Superiore}{30 Luglio 2021}
		{Liceo Scientifico "Bonghi-Rosmini", Lucera (FG)}{} \vspace{0.25em}
		\nbItListStart \myItem{Materie principali trattate:} \itemListEnd
		\itemListStart
		\myItem{Matematica}
		\myItem{Fisica}
		\myItem{Biologia}
		\myItem{Chimica}
		\itemListEnd
	}
	
	% -------------------- PROJECTS --------------------
	\sectionBlock{
		\section{Esperienze Accademiche}
	}{
		\projHeading
		{\fontsize{13pt}{13pt}\textbf{UniMove - Car Sharing per Studenti Universitari}}
		{Università degli Studi del Molise, Pesche (IS)} {03/2025 - 09/2025}
		{In questo progetto, ho assunto il ruolo di \textbf{Lead Software Manager}.
			Ho tenuto sessioni di allenamento per gli studenti dell'anno accademico precedente che erano responsabili di sviluppare l'applicazione mobile. I corsi coprivano i fondamentali delle operazioni CRUD, PostgreSQL, Java Spring per lo sviluppo backend, testing API con Postman, e il framework Ionic per lo sviluppo frontend.
			Inoltre, insieme agli altri manager, ho coordinato l'assegnazione delle task seguendo la metodologia Agile, monitorando costantemente il progresso del team, e fornendo feedback strutturale durante tutte le fasi di sviluppo per garantire un flusso di lavoro ottimizzato ed efficiente.}
		{}
		
		\projHeading
		{\fontsize{13pt}{13pt}\textbf{GameTracker - Library Manager}}
		{Università degli Studi del Molise, Pesche (IS)} {05/2024 - 07/2024}
		{Un'applicazione mobile (Android e iOS) sviluppata con Flutter per permettere agli utenti di tracciare i propri video giochi su molteplici piattaforme.}
		{Link al progetto: \url{https://github.com/ShyVortex/game-tracker}}
		\itemListStart
		\myItem{Permette agli utenti di registrare la propria libreria videoludica (piattaforma, tempo di gioco, data di completamento, momenti salienti) e gestire lista dei desideri, preferiti e titoli completati.}
		\myItem{Implementati creazione account utente e sincronizzazione con un backend remoto per preservare i dati utenti su più dispositivi.}
		\myItem{Integrata funzionalità di geolocalizzazione tramite Capacitor e OpenStreetMap per indicare dove un gioco è stato completato (luogo fisico).}
		\myItem{Costruito un menù impostazioni che include il theme switching (modalità chiara/scura) e la personalizzazione del profilo.}
		\myItem{Coordinato lo sviluppo sia del frontend (Flutter) che le API del backend per garantire un'esperienza responsiva su tutte le tipologie di dispositivi mobile.}
		\itemListEnd
		\vspace{0.65em}
		
		\projHeading
		{\fontsize{13pt}{13pt}\textbf{PlusOne - Comunicazione Sanitaria Semplice}}
		{Università degli Studi del Molise, Pesche (IS)} {04/2024 - 06/2024}
		{Ho contribuito allo sviluppo di un'applicazione mobile cross-platform utilizzando il framework Ionic per potenziare la comunicazione tra professionisti sanitari negli ospedali e nei pronto soccorso. L'app linearizza le interazioni tra dottori, pazienti e infermieri, facilitando uno scambio d'informazioni efficiente e migliorando la coordinazione.}
		{Link al progetto: \url{https://github.com/ShyVortex/plusone-ionic}}\vspace{0.65em}
		
		\projHeading
		{\fontsize{13pt}{13pt}\textbf{Diffusion Tool - AI Image Generator and Upscaler}}
		{Università degli Studi del Molise, Pesche (IS)} {12/2023 - 04/2024}
		{Applicazione desktop JavaFX che integra un backend in Python collegato con le pipeline di Stable Diffusion, modelli BSRGAN, e algoritmi di upscaling personalizzati. Progettato per un utilizzo offline, lo strumento offre la generazione di immagini e un upscaling di alta qualità, tutto mantenendo i dati utente in locale.}
		{Link al progetto: \url{https://github.com/ShyVortex/diffusion-tool}}
		\itemListStart
		\myItem{Strutturata l'interfaccia grafica con JavaFX e gestita l'integrazione con Python per i flussi di lavoro con IA.}
		\myItem{Implementati i flussi di autenticazione per l'utente (Login / Sign Up) e gestita la navigazione ai moduli Home, Profile, Generate, e Upscale.}
		\myItem{Definiti i requisiti di sistema e la gestione delle dipendenze (OpenJDK 17+, Maven, Python + ambiente virtuale)}
		\myItem{Impacchettato il progetto come file .jar eseguibile con build automatizzata (Maven) per una facile distribuzione.}
		\itemListEnd
		\vspace{0.65em}
		
		\projHeading
		{\fontsize{13pt}{13pt}\textbf{Simplified Monopoly}}
		{Università degli Studi del Molise, Pesche (IS)} {08/2022 - 10/2023}
		{Applicazione Java che converte una versione CLI di Monopoly in una con interfaccia grafica utilizzando Swing / GUI Designer, senza alterare la logica di gioco.}
		{Link al progetto: \url{https://github.com/ShyVortex/simplified-monopoly}}
		\itemListStart
		\myItem{Progettata e implementata l'interfaccia utente tramite GUI Designer e JFormDesigner, inclusi il layout, i controlli, e le finestre di gioco.}
		\myItem{Gestita la build del progetto e la distribuzione con Maven e JDK 17.}
		\myItem{Ho imparato i principi di programmazione OOP in maniera intensiva, rifinendo la struttura del codice, e migliorando i design pattern per una manutenibilità futura.}
		\itemListEnd
	}
	
	% -------------------- CERTIFICATIONS --------------------
	\sectionBlock{
		\section{Certificazioni}
	}{
		\eduHeading
		{\large \href{https://github.com/ShyVortex/angelo-resume/blob/main/Certs/B2Cert.pdf}{First Certificate in English} (CEFR Level B2)}{10 Maggio 2025}
		{Cambridge English}{} \vspace{0.25em}
		\nbItListStart
		\myItem{Grado: B}
		\myItem{Punteggio Complessivo: 179}
		\myItem{Centro di riferimento: IT955 0125}
		\myItem{ID di Verifica: D1233928}
		\myItem{ID di Accreditazione: 500/2705/0}
		\itemListEnd
		\vspace{0.65em}
		
		\eduHeading
		{\large Getting Started with Rust (LFEL1002)}{14 Dicembre 2024}
		{The Linux Foundation}{} \vspace{0.25em}
		\nbItListStart
		\myItem{ID Certificazione: LF-4evagqbkz5}
		\myItem{Link di verifica: \href{https://www.credly.com/badges/a24c6224-59d0-4555-8885-e2d83ee7ddd8/public\_url}{Credly}}
		\itemListEnd
		\vspace{0.65em}
		
		\eduHeading
		{\large \href{https://github.com/ShyVortex/angelo-resume/blob/main/Certs/EIPASS.pdf}{EIPASS 7 Moduli User}}{27 Febbraio 2020}
		{Certipass}{} \vspace{0.25em}
		\nbItListStart
		\myItem{Livello: Intermedio}
		\myItem{Codice Utente: EIC00376111AA}
		\myItem{ID di Verifica: WRAQPNYT5L}
		\itemListEnd
	}
	
	% -------------------- ARTICLES --------------------
	\sectionBlock{
		\section{Articoli}
	}{
		\pubListStart
		
		\justifying
		
		\item \textbf{Angelo Trotta}, Fabrizio Perrone. Tecniche di Steganografia e Analisi Forense delle Immagini Digitali. \textit{University of Molise, Pesche, Italy}, 2024. \url{https://github.com/ShyVortex/image-forensics/releases/download/article-1.0/article.pdf}
		
		\pubListEnd
	}
	
	% -------------------- SKILLS --------------------
	\sectionBlock{
		\section{Competenze Personali}
	}{
		\skillListStart
		\justifying
		\item \emph{Lingue parlate}: Italiano, Inglese.
		\item \emph{Linguaggi di programmazione}:
		{\Courier \fontsize{11pt}{11pt}\selectfont Java},
		{\Courier \fontsize{11pt}{11pt}\selectfont C},
		{\Courier \fontsize{11pt}{11pt}\selectfont C\#},
		{\Courier \fontsize{11pt}{11pt}\selectfont Python},
		{\Courier \fontsize{11pt}{11pt}\selectfont TypeScript},
		{\Courier \fontsize{11pt}{11pt}\selectfont Dart},
		{\Courier \fontsize{11pt}{11pt}\selectfont Rust},
		{\Courier \fontsize{11pt}{11pt}\selectfont HTML \& CSS}.
		\skillListEnd
	}\vspace{4pt}
	
	\setstretch{1.1}
	
	\centering
	\renewcommand{\arraystretch}{1.3}
	\begin{tabularx}{\textwidth}{lcccccc}
		\multicolumn{6}{l}{\textbf{{Competenze Linguistiche}}} \\[4pt]
		\toprule
		& \multicolumn{2}{c}{\CronosSbd{COMPRENSIONE}} & \multicolumn{2}{c}{\CronosSbd{PARLATO}} & \multicolumn{1}{c}{\CronosSbd{PRODUZIONE SCRITTA}} \\[3pt]
		\cmidrule(lr){2-3} \cmidrule(lr){4-5}
		& \textbf{Ascolto} & \textbf{Lettura} & \textbf{Interazione} & \textbf{Produzione orale} \\
		\midrule
		\textbf{Italiano} & C2 & C2 & C2 & C2 & C2 \\
		\multicolumn{6}{l}{\textit{Percorso educativo effettuato in Italia.}} \\[3pt]
		\textbf{Inglese} & B2 & B2 & B2 & B2 & B2 \\
		\multicolumn{6}{l}{\textit{Ottenuta la certificazione Cambridge English.}} \\
		\bottomrule
	\end{tabularx}
	
	\vspace{4pt}
	{\footnotesize
		Livelli: A1/A2: Utente base – B1/B2: Utente intermedio – C1/C2: Utente avanzato \par
		\href{https://europass.europa.eu/it/common-european-framework-reference-language-skills}{Quadro Comune Europeo di Riferimento delle Lingue}
	}\vspace{4pt}
	
	\begin{table}[h!]
		\centering
		\renewcommand\arraystretch{1.5}
		\renewcommand\tabularxcolumn[1]{>{\centering\arraybackslash}m{#1}}
		\setlength{\tabcolsep}{6pt}
		\begin{tabularx}{\textwidth}{l*{5}{X}}
			\multicolumn{6}{l}{\textbf{{Competenze Digitali}}} \\[4pt]
			\toprule
			& \multicolumn{5}{c}{\textbf{{AUTOVALUTAZIONE}}} \\
			\hdashline
			& {\textbf{Elaborazione delle informazioni}}
			& {\textbf{Comunicazione}}
			& {\textbf{Creazione di contenuti}}
			& {\textbf{Sicurezza}}
			& {\textbf{Risoluzione di problemi}} \\
			\midrule
			\textbf{Pacchetto Office} & Utente Avanzato & Utente Avanzato & Utente Avanzato & Utente Avanzato & Utente Avanzato \\
			\textbf{Editing foto \& video} & Utente Avanzato & Utente Avanzato & Utente Avanzato & Utente Avanzato & Utente Avanzato \\
			\bottomrule
		\end{tabularx}
		
		\vspace{4pt}
		{\footnotesize
			{Livelli: Utente base – Utente intermedio – Utente avanzato} \par
			\href{https://www.cittadinanzadigitale.eu/wp-content/uploads/2015/10/Competenze-digitali.pdf}{Competenze digitali - Scheda per l'autovalutazione}
		}
	\end{table}
	
	\skillListStart
	\justifying
	\item \emph{Altre competenze}: Tecnico hardware ed entusiasta di modding PC
	\item \emph{Patente di guida}: B
	\skillListEnd
	
	\vspace{1cm}
	
	
	\begin{tabularx}{\textwidth}{>{\centering\arraybackslash}X >{\centering\arraybackslash}X}
		{\textbf{Data}} &
		{\textbf{Firma}} \\[0.3cm]
		\rule{4cm}{0.3pt} & \rule{4cm}{0.3pt} \\[-0.1cm]
	\end{tabularx}
	
\end{document}

